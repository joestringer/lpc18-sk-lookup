\documentclass[10pt,sigconf,authorversion]{lpc}
\usepackage{balance}
\usepackage{courier}
\usepackage{helvet}
\usepackage[utf8]{inputenc}
\usepackage[T1]{fontenc}
\usepackage{listings}
\usepackage{times}
\usepackage{xcolor}

\pdfinfo{
/Title (Building socket-aware BPF programs)
/Author (Joe Stringer)}

\title{Building socket-aware BPF programs}
\author{Joe Stringer}
\affiliation{%
      \institution{Cilium.io}}
\email{joe@cilium.io}

\acmDOI{}
\setcounter{secnumdepth}{0}
\setcopyright{none}

\usepackage{xcolor}
\newcommand\todo[1]{\textcolor{red}{#1}}

\begin{filecontents*}{sk-lookup-api.c}
\import{sk-lookup-api.c}
\end{filecontents*}
\begin{filecontents*}{sk-lookup-structures.c}
\import{sk-lookup-structures.c}
\end{filecontents*}

\begin{document}

\begin{abstract}

    Over the past several years, BPF has steadily become more powerful in
    multiple ways: Through building more intelligence into the verifier which
    allows more complex programs to be loaded, and through extension of the API
    such as by adding new map types and new native BPF function calls. While
    BPF has its roots in applying filters at the socket layer, the ability to
    introspect the sockets relating to traffic being filtered has been limited.

    To build such awareness into a BPF helper, the verifier needs the ability
    to track the safety of the calls, including appropriate reference counting
    upon the underlying socket. This paper describes extensions to the verifier
    to perform tracking of references in a BPF program. This allows BPF
    developers to extend the UAPI with functions that allocate and release
    resources within the execution lifetime of a BPF program, and the verifier
    will validate that the resources are released exactly once prior to program
    completion.

    Using this new reference tracking ability in the verifier, we add socket
    lookup and release function calls to the BPF API, allowing BPF programs to
    safely find a socket and build logic upon the presence or attributes of a
    socket. This can be used to load-balance traffic based on the presence of a
    listening application, or to implement stateful firewalling primitives to
    understand whether traffic for this connection has been seen before. With
    this new functionality, BPF programs can integrate more closely with the
    networking stack's understanding of the traffic transiting the kernel.

\end{abstract}

\maketitle

\section{Keywords}

BPF, Linux, packet processing, sockets

\section{Introduction}

\begin{itemize}
\item BPF started with sockets, but only provided access to packets
\item Previous helper extensions allowed e.g. socket operation handling
\item Arbitrary socket lookup from socket/packet hooks allows more powerful BPF
      programs
\item Use cases revolve around building logic in socket/packet layer hooks that
      change the behaviour based on the existence and attributes of a socket.
\end{itemize}

\section{Extending the Verifier}

%To validate access around sockets, the verifier must have knowledge of the
%socket type to validate that access is within bounds and is safe with respect
%to read or write.

The BPF verifier must ensure that all BPF programs that are loaded into the
kernel are safe to execute. For accessing socket pointers, this requires
validation that memory accesses are within the bounds of the memory allocated
for the structure, and will not make any modifications to the socket fields
that would be incompatible with the core socket handling logic. This is
provided through extension of existing functionality which performs bounds
verification and offset conversion for programs operating on CGroups hooks.
Additionally, the underlying memory must remain associated with the socket for
the duration of accesses from the BPF program, which may be guaranteed by
taking a reference on sockets for the duration of their use. The following
sections explain how this is implemented through the introduction of a new
pointer type in the verifier, and through the extension of the verifier to
recognise assembly instructions that represent reference acquisition and
release.

\subsection{Pointers to Sockets}

Earlier work~\cite{bpf-sock} introduced a \verb+bpf_sock+ structure to the BPF
API and implemented bounds and access type checking, along with offset
rewriting for converting the access of socket attributes from BPF instructions
into equivalent accesses of the underlying kernel types. This work was
introduced with new BPF hook points which provide the socket structure as the
context of the BPF program.

To allow BPF programs to retrieve and access a socket pointer, the verifier
must be made aware of when a register contains a pointer of this type, and also
how to validate pointer access. The existing verifier logic handles context
pointers in a generic manner via \verb+bpf_verifier_ops+. However, the packet
event hook points targeted in this paper already contain a context pointer of a
type that is different from \verb+bpf_sock+, so some additional logic was
required in the verifier to understand this pointer type. The patch series
associated with this paper introduced a new pointer type specific to sockets,
and linked its validation directly to the aforementioned socket verification
functions without using the verifier operations abstraction. Write operations
into the socket are rejected in the current implementation.

With the verifier now aware of socket pointer types, this could be built upon
to implicitly associate socket pointers with references to kernel resources.

\subsection{Reference tracking}

Sockets in Linux are reference-counted to track usage and handle memory
release. In general to ensure the safety of socket for the duration of access
in BPF programs, this reference count needs to be incremented when sockets are
referenced from BPF programs. Two options were considered for how to guarantee
this: Implicit reference tracking and explicit reference tracking.

%To ensure that the socket remains safe to access for a period of BPF program
%execution, the BPF program must also take a reference on the socket. This also
%implies that such references must be subsequently released to prevent leaking
%kernel resources.

\subsubsection{Implicit reference tracking.}

This calls for the BPF infrastructure to handle reference tracking, and to hide
this detail from the BPF API. Whenever a socket lookup function is called, a
reference is taken and the socket is added to a reference list. At the end of
the BPF program execution, the core code could walk this reference list and
release each reference. However, collecting references and releasing them at
the end of the BPF program invocation has the unfortunate overhead that even
the execution of BPF programs that do not make use of the new socket lookup
helpers would need to pay the cost of checking the list of references to free.
When dealing with sufficiently demanding use cases such as those which use
eXpress Data Path (XDP)~\cite{xdp}, even the few instructions required to
implement such a check may have a measurable impact on performance.

\subsubsection{Explicit reference tracking.}

This calls for reference acquisition and release semantics to be built into the
API. If references must be taken, then requiring BPF program writers to
explicitly handle these references ensures that the program writer must
understand the potential impact this may have on the operation of the program
(including atomic operation cost), and it also ensures that the cost is only
borne by programs that make use of this feature.

\subsubsection{Implementation.}

Based upon the tradeoffs described above, explicit reference tracking was
chosen. From an implementation perspective, this requires tracking pointer
values through each conditional path in the program, and rejecting the load of
programs that fail to "balance" resource acquisition (lookup) with release.
This implementation works as follows.

Calls to helper functions which acquire references to resources are annotated
in the verifier to associate the acquisition and release functions. When the
instruction that acquires a resource is processed, a resource identifier is
allocated for this resource. This identifier is kept in a list in the verifier
state, and it is also associated with the register which receives the resulting
resource pointer. When processing a corresponding release helper call, if a
parameter to the function contains a register that is associated with a
resource identifier, then the identifier is removed from the verifier state. If
any paths in the program reach the final (\verb+BPF_EXIT+) instruction and the
verifier state contains any resource identifier association, then the program
is considered unsafe as it has leaked the resource.

Some instruction calls are restricted while holding a resource reference to
avoid leaking references, for example the \verb+bpf_tail_call+ helper function,
which would otherwise leak the pointer reference to a subsequent BPF program.

\subsection{Runtime protection}

The verifier is tasked with ensuring that all BPF program execution flow is
safe to the extents where it interacts with the BPF API. In general, BPF
programs, once verified and triggered by event hooks, will execute from the
first instruction through the final instruction for a particular path through
the program. Therefore, after execution of a BPF program containing resource
acquire and release functions which are verified by the logic above, references
to those resources should not leak.

One exception to this is when classic BPF \verb+LD_ABS+ or \verb+LD_IND+
instructions are used in the BPF program. These instructions were previously
used to provide direct packet access, with the semantics that if the access
offset exceeds the length of the packet, the BPF program would be terminated
prior to the final state. Without mitigation, this could lead to leaking of
references at runtime, even if the instructions appear to balance acquisition
and release at verification time. Newer BPF APIs provide better alternatives to
these instructions for packet access, so we disallow the use of these
instructions while holding a socket reference.

\section{Extending the BPF API}

To provide an API that implements explicit reference tracking, at least two
functions are needed to handle referencing of sockets: A socket lookup function
and a socket release function. This section describes the API considerations
for each of these.

\subsection{API definition}

The following considerations were made for implementing the lookup helpers. The
consideration of each of these items reflects the definition of the lookup
helpers in Listing~\ref{lst:helper-api} and the structure definitions in
Listing~\ref{lst:helper-struct}.

\begin{itemize}
    \item The Linux stack may contain multiple network namespaces, where the
          BPF program may operate attached to a device in one network namespace
          while the desired socket may exist in another network namespace.
    \item BPF programmers may wish to discover sockets that are not directly
          related to the BPF program context (packet, socket, etc).
    \item Current use cases are focused around UDP and TCP sockets, however
          there may be a desire to extend this in future.
\end{itemize}

\lstinputlisting[float=*b,%
  caption={BPF API helper functions for socket lookup},%
  language=c,%
  label=lst:helper-api,%
  breaklines=true]{sk-lookup-api.c}

\lstinputlisting[float=*b,%
  caption={BPF API structures for socket lookup (Linux v4.20)},%
  language=c,%
  label=lst:helper-struct,%
  breaklines=true]{sk-lookup-structures.c}

\subsubsection{Namespacing.}

Linux network devices are associated with a particular network namespace which
provides logical network stack separation~\cite{netns, netns-netdevice}.
Sockets are also inherently associated with an application that operates in a
particular network namespace~\cite{netns-sock}. The simplest path for handling
namespacing from socket lookup handlers operating from a network device would
be to allow BPF programs operating on a device within a network namespace to
only find sockets in the same network namespace. By passing the BPF program
context into the socket lookup helper, the implementation can derive the source
network namespace from this context. Beyond this, with the growing popularity
of containers it may be desirable for a network orchestration tool to make use
of socket properties inside container network namespaces to influence decisions
made on packets outside the container---for instance, on packets being received
into a node prior to passing the packet into the network namespace. By allowing
the netns ID to be specified, this use case can be supported as well.

\subsubsection{Tuple definition.}

A common use case for the socket lookup functions would call them from a packet
context, where the \verb+__sk_buff+ has the full 5-tuple information available
and easily accessible. One could imagine a simpler helper function that allows
the BPF program writer to provide the \verb+__sk_buff+ to the helper function,
then the helper function would look up the socket based upon the packet
metadata. This would however limit the potential uses for such a helper. Two
cases that would be more limited with this model are usage from XDP (which does
not inherently parse packets or collect packet metadata into the context), or
if the BPF program implements a form of network address translation. From XDP,
if the raw \verb+xdp_md+ context is provided to such a helper, then the helper
would need to parse the packet to understand the 5-tuple. This would duplicate
the standard stack paths which are executed after XDP where the \verb+sk_buff+
is built, but this information would be subsequently thrown away upon return of
the helper call. Adding to this, if the program performs network address
translation between the local application and the remote destination, then the
local stack may not contain a socket associated with the tuple that is directly
in the packet. While these cases could be worked around with a simpler helper,
it was deemed more powerful and generic to allow the BPF program writer to
provide the tuple for lookup.

\subsubsection{Extensibility.}

New kernel APIs that have any scope for extension should contain a flags
argument. For the socket lookup helpers, a few ideas had been considered as
potential future alternatives to the existing behaviour. The lookup functions
follow the standard socket lookup paths in Linux which have predetermined
methods for selecting a socket when the application uses \verb+SO_REUSEPORT+.
Some subsequent discussion on the mailinglist proposed allowing BPF program
writers to influence the socket selection mechanism~\cite{netdev-sk-select}.

\subsubsection{Result.}

Another aspect of extensibility is the ability to lookup sockets which are not
TCP or UDP. The initial RFC of this patch series proposed a single lookup
function which would choose the Layer~4 protocol based upon a field in the
tuple~\cite{sk-lookup-rfc}, however this would make it more difficult for BPF
program writers to detect the support for different protocol types at compile
time. When a socket type is unsupported, the implementation would return
\verb+NULL+, implying that there is no such socket. There is not a significant
number of supported Layer~4 socket types in the Linux stack today, and the
number is not expected to increase drastically, so it was considered simpler
and easier to split out each Layer~4 socket lookup function into an independent
helper call, so that the result could simply return a socket or \verb+NULL+ and
not need to encode a representation of other error conditions.

\subsection{Optimizations}

Multiple optimizations were proposed during development of this feature to
reduce the runtime overhead of using the socket lookup helpers. Two such
optimizations are described below: Skipping reference counting when unnecessary
to save on atomic instructions, and allowing the use of direct packet pointers
for the tuple.

\subsubsection{Avoid reference counting when unnecessary.}

For some socket types, such as UDP sockets, or TCP listen sockets, memory
access safety can be achieved with minimal implementation: The destruction of
such sockets is already governed by standard RCU rules, meaning that while the
RCU lock is held, they can be safely accessed without holding a reference; Once
the RCU grace period is reached, the memory may be freed and references to the
socket are no longer safe to use. For these socket types, since BPF programs
run under the RCU lock, the properties of these sockets can be accessed
directly without taking a reference on the socket. As such, the implementation
of the socket lookup and release functions can avoid the atomic reference count
increment operation.

The API retains explicit reference tracking to ensure consistency of the API
and to allow multiple underlying implementations to handle reference tracking
in a way that is safe for the implementation. For instance, for  TCP sockets
that are not governed by RCU, the networking stack uses reference counting to
manage socket memory instead, so the socket lookup and release functions would
take and release references on the socket, respectively.

\subsubsection{Allow lookup using direct packet pointers.}

The tuple structure is defined in such a way that if the IPv4 or IPv6 packet is
immediately followed by the TCP or UDP header without IP options in between,
then a pointer to the packet data at the offset of the Layer~3 addresses may be
passed to the lookup function, allowing the implementation to directly pull the
addresses and ports from the packet buffer rather than requiring the BPF
program to first extract these onto the stack and pass a pointer to the stack
copy of the tuple.

\subsection{Future work}

The implementation introduced in the 4.20 release cycle includes support for
looking up TCP and UDP sockets for IPv4 and IPv6 traffic, using the
\verb+SCHED_CLS+, \verb+SCHED_ACT+ and \verb+SK_SKB+ hook points. A proposed
patchset extends this to allow the same helpers to be used from the \verb+XDP+
hook points~\cite{sk-lookup-xdp}.

\todo{Associate custom metadata with socket?}

\section{Use cases}

\subsection{Stateful Firewalling}

Stateful firewalling is distinguished from stateless firewalling in that it
provides the ability to associate two directions of a connection together and
apply a filtering policy based upon the direction of the connection. This
commonly involves using a connection tracker to reconstruct the connection
state of each endpoint of the connection without specific knowledge from those
endpoints~\cite{conntrack,ovs-ct}. In cases where Linux is configured as a
firewall between applications that exist in distinct Linux kernel copies
running in the network, this makes sense; The information that is needed to
perform stateful firewalling is not present on the Linux instance that needs
this information, so the information must either be shared from the peers
(assuming that the firewall orchestrator has control over the peer endpoints),
or this information must be synthesised based upon the packets that are seen on
this intermediate Linux instance. However, in many cases this model is used
when filtering traffic where one endpoint is co-located with the Linux stack
that performs the firewalling, which already contains information about the
local end of the connection. In this use case, the connection tracking table
effectively duplicates knowledge that Linux already creates and maintains.

Using the socket lookup functions described in this paper, in conjunction with
the socket properties including local Layer~3 and Layer~4 information available
in the \verb+bpf_sock+ (Listing~\ref{lst:helper-struct}), BPF programs can
identify the directionality of connections for packets ingressing or egressing
the Linux stack on a network device.

\subsection{Load-balancing}

\section{Conclusion}

This paper describes the contribution of new BPF verifier functionality in the
Linux kernel to track references to kernel resources and uses these to provide
access to socket introspection capabilities from packet handling hooks. The
introduction of reference tracking logic into the verifier provides
useful base infrastructure for supporting acquire and release semantics for
kernel resources which may prove useful for other helpers in future, and the
socket lookup API makes the packet hooks more powerful to support use cases
such as stateful firewalling or load-balancing based on the sockets that are
open on a system.

\section{Acknowledgments}

The core BPF maintainers provided great guidance and review of the patches, in
particular Alexei Starovoitov, Daniel Borkmann and Martin KaFai Lau. Additional
shout-out to Nitin Hande who provided testing and feedback.

\bibliographystyle{plainnat}
\bibliography{paper}

\end{document}

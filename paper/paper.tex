\documentclass[10pt,sigconf,authorversion]{lpc}
\usepackage{balance}
\usepackage{courier}
\usepackage{helvet}
\usepackage[utf8]{inputenc}
\usepackage[T1]{fontenc}
\usepackage{listings}
\usepackage{times}
\usepackage{xcolor}

\pdfinfo{
/Title (Building socket-aware BPF programs)
/Author (Joe Stringer)}

\title{Building socket-aware BPF programs}
\author{Joe Stringer}
\affiliation{%
      \institution{Cilium.io}}
\email{joe@cilium.io}

\acmDOI{}
\setcounter{secnumdepth}{0}
\setcopyright{none}

\usepackage{xcolor}
\newcommand\todo[1]{\textcolor{red}{#1}}

\begin{document}

\begin{abstract}

Over the past several years, BPF has steadily become more powerful in
multiple ways: Through building more intelligence into the verifier which
allows more complex programs to be loaded, and through extension of the API
such as by adding new map types and new native BPF function calls. While
BPF has its roots in applying filters at the socket layer, the ability to
introspect the sockets relating to traffic being filtered has been limited.

To build such awareness into a BPF helper, the verifier needs the ability to
track the safety of the calls, including appropriate reference counting upon
the underlying socket. This talk walks through extensions to the verifier to
perform tracking of references in a BPF program. This allows BPF developers to
extend the UAPI with functions that allocate and release resources within the
execution lifetime of a BPF program, and the verifier will validate that the
resources are released exactly once prior to program completion.

Using this new reference tracking ability in the verifier, we add socket lookup
and release function calls to the BPF API, allowing BPF programs to safely find
a socket and build logic upon the presence or attributes of a socket. This can
be used to load-balance traffic based on the presence of a listening
application, or to implement stateful firewalling primitives to understand
whether traffic for this connection has been seen before. With this new
functionality, BPF programs can integrate more closely with the networking
stack's understanding of the traffic transiting the kernel.

\end{abstract}

\maketitle

\section{Keywords}

BPF, Linux, packet processing, sockets

\section{Introduction}

\section{Acknowledgments}


\bibliographystyle{plainnat}
\bibliography{paper}

\end{document}
